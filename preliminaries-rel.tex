

\section{Introduction}
\label{sec:introduction}

Let $\gls{V}$ be a complex vector space, and let $\gls{G}$ be a connected reductive
algebraic group with a fixed faithful linear action on $V$. 
Attached to this data, we have a symplectic variety $\gls{Coulomb}$ called the {\bf
  Coulomb branch}, defined by Braverman, Finkelberg and Nakajima \cite{BFN},
based on proposals in the physics literature. Many interesting
varieties appear this way, including quiver varieties in finite and
affine type A, hypertoric varieties and slices between Schubert cells in affine Grassmannians.  Part of the BFN construction is
the construction of a number of partial resolutions $\gls{tM}$ of $\fM$; we'll
call one of these a {\bf BFN resolution} if it is a resolution of singularities.

Bezrukavnikov and Kaledin have developed a general theory of quantizations
of algebraic varieties in arbitrary characteristic
\cite{BK04a,BKpos} and Kaledin showed that this theory can
be applied to construct tilting generators on symplectic resolutions
of singularities \cite{KalDEQ}.  Kaledin's theory is very powerful,
but not very concrete from the perspective of a representation theorist.
In particular, this work shows that the category of coherent sheaves on a conic
symplectic resolution is derived equivalent to the category of modules
over an algebra $A$ (actually to many different algebras, one for each 
choice of a quantization parameter), but in any particular case, this algebra is quite
challenging to calculate.  Our goal in this paper is to develop
Kaledin's theory as explicitly as possible in the case of Coulomb
branches and in particular to describe this algebra $A$.  We will show:
\begin{itheorem}\label{th:main}
Any BFN Coulomb branch with a BFN resolution has an explicit
combinatorially presented non-commutative resolution of singularities
$\gls{A}$.  The category $D^b(A\mmod)$ is  equivalent to the derived category of coherent sheaves on any BFN resolution via an explicit tilting generator.
\end{itheorem}
For readers who prefer to live in characteristic 0 to characteristic
$p$, we should emphasize that the construction of this non-commutative
resolution $A$ and its tilting generator have a construction which is
characteristic free (that is, over $\Z$);  however, we use reduction
to characteristic $p$ and comparison to the Bezrukavnikov-Kaledin
method to confirm Theorem \ref{th:main}.

Of course, this theorem is only of interest to a reader who knows some
examples of Coulomb branches with BFN resolutions.  Most interesting
come from a quiver gauge theory, that is,
the case which leads to Nakajima quiver varieties as Higgs branches.
Following Nakajima's notation, for a quiver $\Gamma$ with vertex set $\vertex$,
consider dimension vectors $\Bv,\Bw\colon \vertex\to \Z_{\geq 0}$, and
the group and representation
\begin{equation}
\gls{G}=\prod GL(\C^{v_i})\qquad \gls{V}=\Big(\bigoplus_{i\to j}
\Hom(\C^{v_i},\C^{v_j})\Big)\bigoplus \Big(\bigoplus_{i\in {\vertex}}
  \Hom(\C^{w_i},\C^{v_i}) \Big),\label{eq:quiver-gauge}
\end{equation}
  with the obvious induced action.  In this
case, the algebra $A$ is a version of a KLR algebra drawn on a
cylinder, as we will show in the second part of this paper \cite{WebcohII}.  Note
that:
\begin{enumerate}
\item When the underlying quiver is of type A, then the resulting
  Coulomb branch is the Slodowy slice to one nilpotent orbit inside
  another in a type A nilcone.  The BFN resolutions in this case are
  exactly those which arise from taking the preimage under a
  resolution of the larger orbit closure by $T^*(SL_n/P)$ for a
  parabolic $P$. 
\item When the underlying quiver is of type D or E, the Coulomb branch is isomorphic to an affine Grassmannian slice, as shown in
  \cite[App. B]{BFNplus}.
\item When the underlying quiver is a loop, the resulting Coulomb
  branch is the $v$-fold symmetric product of the singular surface
  $S=\C^2/(\Z/w\Z)$.  In particular, one of the BFN resolutions we obtain
 is the Hilbert scheme of $v$ points on the crepant resolution $\tilde
 S$.  
\item When the underlying graph is an $n$-cycle, we
  obtain a Nakajima quiver variety (or more generally a bow variety) for a cycle of size $w=w_i$ whose
  dimension vectors are related to $\Bv,\Bw$ by a version of
  rank-level duality \cite{NTbow}, including both the results
  above as special cases.
\end{enumerate}

The algebra $\gls{A}$,
which appears as endomorphisms of this tilting generator, can be
interpreted in three very interesting ways:
\begin{enumerate}
\item it can be described algebraically  as a finitely generated algebra constructed directly from the combinatorics of the group $G$ and representation $V$.
\item it can also be described as a convolution algebra in the
  extended BFN category of \cite{WebSD}, with adjusted flavor and $h=0$ (we
  call these ``\gls{pthroot} conventions.'').
\item it is the endomorphism algebra of a finite sum of line
  defects in the corresponding $\mathcal{N}=4$ supersymmetric $3d$
  gauge theory (see \cite{DGGH} for a more thorough discussion of this
  category).
\end{enumerate}
The equivalence of these descriptions is discussed in \cite{WebSD}:
the equivalence of (1) and (2) is \cite[Th. 3.11]{WebSD} and of (2)
and (3) is\cite[Rem. 3.6]{WebSD};  the latter is a motivational
statement rather than a theorem since we are not working with a
precise definition of the category of line defects.  We expect this
will be remedied in \cite{DGGH}.

\subsection{Motivation from line defects}
\label{sec:motivation-from-line}

Before getting bogged down in details, let us try to give a general
sketch of our approach.  In this section, we play a little fast and
loose with the existence of certain geometric categories (not to
mention aspects of quantum field theory); we promise to the reader
that no such chicanery will appear in the rest of the paper.

Having fixed the vector space $V$ and gauge
group $G$, we consider the quotient space of $\C((t))$-points
$\mathcal{L}=V((t))/G((t))$.  In the world of derived algebraic geometry, we think
of this as the loop space of the stack quotient $V/G$.  We will
consider the category of $D$-modules on this space.

  For the utility of both readers who  wish to read
  not-entirely-rigorous physics motivation and for those who wish to avoid it,
  the author will put such motivating paragraphs in ``Physics
  Motivation'' environments going forward.
 \begin{physics}
  This category is natural to consider in this context because it
  should give the category of line defects in a 3-dimension
  $\mathcal{N}=4$ supersymmetric field theory defined by the $A$-twisted
  $\sigma$-model into the $2$-shifted stacky cotangent bundle
  $T^*[2](V/G)$, turned into a 3-dimensional field theory using the
  AKSZ formalism.  In particular, the local operators on a point
  should appear in this category as operators from the trivial line
  defect to itself.
\end{physics}

Obviously, there
are many technical issues involved in doing this, and, with apologies
to the reader, we will make no attempt to resolve them.
We simply
ask the reader to accept the existence of this category as a black box
with one basic property:
\begin{enumerate}
\item Given a reasonable map $p\colon \mathcal{M}\to \mathcal{L}$, we
  have the pushforward of the function D-module
  $\mathscr{O}_p=p_*\mathcal{O}_{\mathcal{M}}$.  If we are given a second
  such map $p'\colon \mathcal{M}'\to \mathcal{L}$ then 
  \[\Ext^\bullet(\mathscr{O}_p,
   \mathscr{O}_{p'})=H^{BM}_*(\mathcal{M}\times_{\mathcal{L}}\mathcal{M'})\]
 with composition induced by convolution.  
\end{enumerate}
The definition of Braverman-Finkelberg-Nakajima is that the functions
on the Coulomb branch  of the gauge theory of $(G,V)$ arise when we
take $\mathcal{M}=\mathcal{M}'=V[[t]]/G[[t]]$, the arc space of the
quotient $V/G$; the natural quantization of this ring
appears when we consider this same construction $\C^*$-equivariantly for the
loop action (where $t$ has weight 1).
\begin{physics}
  As mentioned above, the arc space should give the trivial line, so
  the BFN construction gives the local operators in the $A$-twist of
  the $\sigma$-model discussed, with $\C^*$-equivariance giving the
  $\Omega$-deformed version of these operators (see \cite{NaCoulomb}
  and \cite[\S
  1.3]{BBBDN} for more details).
\end{physics}



However, the utility of this perspective does not stop when we have
contructed the Coulomb branch; given any $\C^*$-action $\nu\colon \C^*
\to \Aut_G(V)$ on $V$
commuting with $G$, we can consider the image $\nu(t^k)\cdot
V[[t]]/G[[t]]$ and the map \[p^{(k)}\colon \nu(t^k)\cdot
V[[t]]/G[[t]]\to V((t))/G((t)).\]  The
BFN resolution $\gls{tM}$ constructed by
Braverman-Finkelberg-Nakajima in \cite{BFNline} can be defined  the
property that \[\Gamma(\tilde{\fM}; \mathcal{O}(k))\cong
  \Ext^\bullet(\mathscr{O}_{p^{(m)}}, \mathscr{O}_{p^{(m+k)}})\] with
multiplication in the projective coordinate ring given by Yoneda
product.  
 Loop equivariantly, these spaces are not isomorphic, but instead give a $\Z$-algebra
in the sense of Gordon and Stafford \cite{GS}; as discussed in
\cite[\S 5.2]{BLPWquant}, this is the quantum homogeneous coordinate
ring of a quantization of $\tilde{M}$.



Thus, given a map  $q\colon \mathcal{M}\to \mathcal{L}$, we can define a coherent
sheaf $\mathcal{Q}_q$ on the BFN resolution.  This is the construction
we require for our tilting generators and noncommutative resolution.
\begin{itheorem}
  There is a space  $\mathcal{M}$ and map $q\colon \mathcal{M}\to \mathcal{L}$ such
  that $ \mathcal{Q}_q$ is the tilting generator of Theorem
  \ref{th:main} and  \[\gls{A}=\Ext^\bullet(\mathscr{O}_{q},
    \mathscr{O}_{q}).\]
\end{itheorem}
We should note that there is not just one such space, but in fact
there are several of them, which give rise to different noncommutative
resolutions and different tilting generators.  These different choices
are related by wall-crossing
functors (as defined, for example, in \cite[\S
2.5.1]{losev2017modular}).  These functors also have a geometric
realization:
\begin{itheorem}
The derived equivalence of coherent sheaves to $A$-modules
intertwines wall-crossing functors with tensor product with the
bimodules $\Ext^\bullet(\mathscr{O}_{q},
    \mathscr{O}_{q'})$ for different spaces $q,q'$ both giving
    resolutions.  These
actions define a Schober in the sense of \cite{KSschobers}, that is,
a perverse sheaf of categories, on a particular subtorus arrangement
in a complex torus.  
\end{itheorem}  
\begin{physics}
  All of these other D-modules can be interpreted naturally in physics
  in terms of natural modifications of the fields of the theory near
  the line defect.  We leave a more detailed discussion of this point
  to future work and the interested reader.

  Unfortunately, the author knows no good physics explanation of
  which combinations of line operators give non-commutative
  resolutions, and which do not.  Obviously, this would be an
  interesting question from a quantum field theory perspective.   
\end{physics}

\subsection{Summary of approach}
\label{sec:summary}

Proving these results depends on comparison with the characteristic
$p$ approach of Bezrukavnikov and Kaledin \cite{BKpos,KalDEQ}.  That
is, we come to understand by considering quantizations in
characteristic $p$ and applying the approach of \cite{WebSD} (based
in turn on \cite{FOD,MVdB}) in
positive characteristic.  The focus is on the
action of a large polynomial subalgebra of the quantum Coulomb branch,
and analyzing the representations of this algebra in terms of their
weights for this subalgebra.  In the perspective of Stadnik \cite{Stadnik} to resolving this problem
for hypertoric varieties,  this polynomial subalgebra played a key
role in constructing the requisite \'etale cover where the Azumaya
algebra constructed from a quantization splits.  

This approach extends to the case of a general BFN Coulomb branch.
Whenever we have a BFN resolution $\gls{tM}$, we obtain an explicit
tilting generator for $\K$ either of large positive characteristic or
characteristic 0, which has a natural description in terms of the BFN
construction.  The sections of its summands (and their twists by ample
line bundles) are given by the homology of spaces on which the
convolution realization of the BFN algebra acts.  
This is an extension of work of McBreen and the author \cite{McBW}, which
shows the same result in the abelian case.

As mentioned above, we'll cover the case of quiver gauge theories in
considerably greater detail in a companion paper \cite{WebcohII}.
Beyond this, there are several interesting possibilities for extension of this
work.  The work of McBreen and author in the abelian case \cite{McBW} can
be used to show one version of homological mirror symmetry for
multiplicative hypertoric varieties, and it would be very interesting
to relate the presentations of $A$ appearing here with the Fukaya
category of multiplicative Coulomb branches (i.e. the algebraic
varieties obtained by the K-theoretic BFN construction).

The tilting bundles that appear also have a natural interpretation in
terms of line operators in the corresponding 3-dimensional gauge
theory, and one could hope that other perspectives on these line
operators, such as the vertex algebra perspective suggested in
Costello, Creuzig and Gaiotto \cite{CCG}, will also see the same combinatorial
constructions appear, hopefully eventually leading to a theory of
S-duality where coherent sheaves on Coulomb branches can be described
as a natural object on the Higgs side as well.

\section{Quantum Coulomb branches}

\subsection{Background}
\label{sec:background}



Let us recall the construction of quantum Coulomb branches from
\cite{WebSD}.  As before, let $\gls{G}$ be a connected reductive algebraic
group over $\C$, with $G((t)), G[[t]]$ its points over
$\C((t)), \C[[t]]$. For a fixed Borel $B\subset G$, we let $\Iwahori$
be the associated Iwahori subgroup
\[\Iwahori=\{g(t)\in G[[t]]\mid g(0)\in B\}\subset G[[t]].\]  The {\bf affine flag variety} $\AF=G((t))/\Iwahori$ is
just the quotient by this Iwahori.  
% The connected components of this
% space are in bijection with the quotient of the coweight lattice of
% $G$ by the coweight lattice of the simply connected cover of $[G,G]$.
% Each component contains a unique fixed point of the Iwahori $I$, given
% by $t^{\mu}$ where $\mu$ is the unique cominuscule coweight in the
% corresponding coset.  


\nc{\wtG}{\widetilde{G((t))}}
Let $\gls{V}$ be the $G$-representation fixed in the previous section, 
$\gls{No}=N_{GL(V)}(G)$ be the normalizer of $G$ in $GL(V)$, and let
$\gls{F}=\No/G$ be the flavor quotient and $\gls{T}_\No,T_{F}$ be compatible maximal tori of
these groups.  
Of course, we use $\gls{ft}_{\No}$, etc. for the Lie algebra of this
torus, $\ft_{\No,S}$ for the subset where integral weights have
values in a subring $S\subset \C$, with the most important cases being
$S=\R$ and $S=\Z$.    It's also useful to consider $\gls{To}$, the preimage of
$T_F$ in $\No$ and $\tilde{\To}=Q\times \C^*$.  This latter groups acts on $V((t))$
such that $vt^a$ has weight $a$ under the second factor and the
obvious action of $\To$.

Fix a flavor $\gls{flav}\colon \C^*\to T_F$, and let 
\[\tilde{G} =\{(g,s)\in \To\times \C^* \mid \flav(s)=g\pmod G\}\qquad
  \nu(g,s)=s,\]
with its induced action on $V((t))$.  
That is, $\tilde{G}$ is the pullback of the diagram $\C^*\to T_F \leftarrow
\To$. Let $\tilde{T}$ be the induced torus of this group, and
$\tilde{\mathfrak{t}}$ its Lie algebra. 

Fix a subspace $U\subset V((t))$ invariant under $\Iwahori$.  
Let $\VB_U:=(G((t)) \times U)/\Iwahori$.  Note that we have a
natural $G((t))$-equivariant projection map $\VB_U\to V((t))$. 
Let
$\wtG$ be the subgroup of $\No((t))\times
\mathbb{C}^*$ generated by $G((t))$ and the image of
$\tilde{G}\hookrightarrow \tilde{G}\rtimes \C^*$ included via the
identity times $\nu$.  

\begin{definition} The BFN Steinberg algebra $\gls{efA}$ is the equivariant Borel-Moore homology group
   \[\gls{efA}=H_*^{BM, \wtG}(\VB_{V[[t]]}\times_{V((t))}\VB_{V[[t]]};\K)\] endowed with the convolution multiplication.  
 \end{definition}
 As discussed in Section \ref{sec:motivation-from-line}, this algebra
is intended to match an Ext algebra in the category of $D$-modules on
$\mathcal{L}$.  We can avoid any technicalites about the nature of
this category by considering this
homology space instead, and interpreting the equivariant homology $H_*^{BM,
  \wtG}(\VB_{V[[t]]}\times_{V((t))}\VB_{V[[t]]};\K)$ using the techniques in \cite[\S 2(ii)]{BFN}. 
  As usual, we let $h$ be
the equivariant parameter corresponding to the character $\nu$, and $\gls{S}_h=H^*(B\tilde{T};\K)=\K[\tilde{\ft}]$, which is naturally a subalgebra of $\EuScript{A}$ under the identification $H^{BM,\wtG}_*(\VB_{V[[t]]})\cong S_h$.  When we specialize $h=0,1$, we will write $S_0,S_1$, etc.


The original BFN algebra \gls{Asph} is
defined in essentially the same way, using
$\EuScript{Y}_{V[[t]]}:=(G((t)) \times V[[t]])/G[[t]]$.  The algebras $\EuScript{A}^{\operatorname{sph}}$ and $\EuScript{A}$
  are Morita equivalent by \cite[Lemma 3.3]{WebSD}, with
  $e_{\operatorname{sph}}
  \EuScript{A}^{\operatorname{sph}}e_{\operatorname{sph}}=\EuScript{A}$
  for an idempotent $e_{\operatorname{sph}}\in \EuScript{A}$.
  \begin{physics}
    In the parlance of quantum field theory, \gls{Asph} is the algebra
    of local operators on a trivial line defect, and $\gls{efA}$ the
    algebra of local operators on the defect that comes from coupling
    to super quantum mechanics on the flag variety $G/B$.  The
    equivariant parameter $h$ corresponds $\Omega$-background for the circle rotating
    around this line in $\R^3$, as discussed in \cite[\S 6]{BBBDN}.
  \end{physics}


  
  
\begin{definition}
The {\bf Coulomb branch} $\gls{Coulomb}$ for $(V,G)$ is the spectrum of the algebra \gls{Asph} after specialization at $h=0$ (at which point it becomes commutative).  The {\bf quantum Coulomb branch} is the specialization of this algebra at $h=1$.
\end{definition}

Of course, $\gls{To}$ still acts on $V$, and thus has an associated Coulomb
branch $\gls{MQ}$.  As discussed in \cite[\S 3]{BFN} and \cite[\S
3.3]{WebSD}, this Coulomb branch has a Hamiltonian action of
$T_F^\vee$, the Langlands dual of the dual of the torus of the flavor
group $\gls{F}$ with moment map given by
$\mathfrak{t}_F^*\to H^*_{\To}(pt)$, and $\fM$ is the categorical
quotient of the zero-level of the moment map on $\fM_{\To}$, the
Coulomb branch for $\To$.  For a given cocharacter of $\gls{T}_F$
(considered as a character of $T_F^\vee$), we can instead take the
associated GIT quotient of $\fM_{\To}$, which gives a variety
$\gls{tM}$ which maps projectively to $\fM$.  As mentioned in the
introduction, if $\tilde{\fM}\to \fM$ is a resolution of singularities
(or equivalently, if $\tilde{\fM}$ is smooth) then we call it a {\bf
  BFN resolution}.

\subsection{The extended category}
\label{sec:extended}
The quantization of the Coulomb branch attached to $(G,V)$ appears as an endomorphism algebra in a larger category, building on the geometric definition of this algebra by Braverman, Finkelberg and Nakajima
\cite{NaCoulomb,BFN}. This category is not unique; there are actually
many variations on it one could choose, and it will be convenient for
us to incorporate a parameter $\delta\in (0,1)\subset \R$ into its
definition; in \cite{WebSD}, we assumed that $\delta=1/2$, but this
played no important role in the results of that paper (in fact, some
results become simpler if we choose $\delta$ generic instead).

Let $\gls{ft1}_{1,\To,\R}\subset \tilde{\ft}_{\To,\R}$ be the preimage
of 1 under projection to $\C=\Lie(\C^*)$, and let $\ft_{1,\R}=\ft_{1,
  \To,\R}\cap\tilde{\ft}$, be the space of real
lifts of the cocharacter $\flav$.

As in \cite{WebSD}, we let
$\{\varphi_i\}$ be the multiset of weights of $V$ (considered as
functions on $\tilde{\ft}_{\To}$)  and we let \[\varphi_i^+=\varphi_i\qquad
\gls{varphimid}^{\operatorname{mid}}_i= (1-\delta) \varphi_i^+-\delta\varphi_i^-=\varphi_i+\delta\nu\qquad \varphi_i^-=-\varphi_i-\nu.\]  

Given any $\acham\in \ft_{1,
  \To,\R}$, we can consider the induced action on the
vector space $V((t))$.  
\begin{itemize}
\item Let $\Iwahori_\acham$ be the subgroup whose Lie algebra is the sum of positive weight
spaces for the adjoint action of $\acham$. This only depends on the
alcove in which $\acham$ lies, i.e. which chamber of the arrangment
given by the hyperplanes $\{\alpha(\acham)=n\mid \al\in \Delta, n\in \Z\}$ contains
$\acham$; the subgroup $\Iwahori_\acham$ is an Iwahori if $\acham$ does not
lie on any of these hyperplanes. 
\item Let  $U_\acham\subset V((t))$ be the subspace of elements of 
weight $\geq -\delta$ under $\acham$.  This subspace is closed under the action of
$\Iwahori_\acham$.  This only depends on the vector $\Ba$ such that
\begin{equation}
\acham\in \gls{AC}_{\Ba}=\{\xi \in \ft_{1,
  \To,\R}\mid a_i<\gls{varphimid}_i^{\operatorname{mid}}(\xi)<a_i+1\text{
  for all $i$}\}.\label{eq:aff-cham}
\end{equation}
\end{itemize}

We call $\acham$ {\bf unexceptional} if does not lie on the unrolled matter hyperplanes
$\{\varphi_i^{\operatorname{mid}}(\acham)=n\mid n\in
\Z\}$ and {\bf generic} if it is unexceptional and does not lie on any
of the unrolled root hyperplanes $\{\alpha(\acham)=n\mid
n\in\Z\}$. We'll call the hyperplanes generic points avoid the {\bf
  unrolled hyperplane arrangment}.  

For any $\acham\in  \ft_{1,
  \To,\R}$, we can
consider
$\VB_{\acham}:=\VB_{U_\acham}:=G((t))\times_{\Iwahori_{\acham}}U_{\acham}$,
the associated vector bundle.  
The space $ \ft_{1,
  \To,\R}$ has a natural adjoint action of
$\gls{What}=N_{\wtG}(\gls{T})/T$, and of course,
$U_{w\cdot \acham}=w\cdot U_{\acham}$.  

We let
\begin{equation}
{}_{\acham}\gls{dVB}_{\acham'}=\left\{(g,v(t))\in G((t))\times
        U_{\acham}\mid g\cdot v(t)\in
        U_{\acham'}\right\}/\Iwahori_{\acham}.\label{eq:aXa}
    \end{equation}
\begin{definition}\label{def:extended-BFN}
  Let the {\bf extended BFN category} $\gls{scrB}^Q$ be the category whose
  objects are unexceptional cocharacters $\acham\in  \gls{ft1}_{1,
  \To,\R}$, with morphisms given by:
  \begin{equation*}
    \Hom(\acham,\acham')=H_*^{BM, \wtG}(\VB_{\acham}\times_{V((t))}\VB_{\acham'};\K)\\
    \cong H_*^{BM, \tilde T}\left({}_{\acham}\VB_{\acham'};
    \K\right).
\end{equation*}
Let $\gls{scrB}$ be the subcategory whose objects are given by $\gls{ft1}_{1,\R}$.
\end{definition}
As before, this homology is defined using the techniques in \cite[\S
2(ii)]{BFN}.
\begin{physics}
  As discussed in Section \ref{sec:motivation-from-line}, the objects
  in this category can be interpreted as D-modules on the loop space
  of $V/G$ (for those inclined toward stacks) or as line defects in a
  $\sigma$-module (for those inclined toward quantum field theory).
  The author has no especially good explanation from either of these
  perspectives why this is the ``right'' subcategory of line operators
  to consider when there are many others available, but it does get the job done.
\end{physics}


Note that by assumption, the
cocharacter $\gls{second}\in \ft_{1,
  \To,\R} $ including $\C^*$ into $\To\times
\C^*$ is unexceptional, but not generic and  $U_{\second}=V[[t]]$.
Given any unexceptional point $\acham$, it has a neighborhood in the
classical topology, which necessarily contains a generic point, on
which $U_{\acham'}=U_{\acham}$. Thus we can find a generic element $\zero$ of the fundamental alcove such that $U_\zero=V[[t]]$. In this case, we
have that $\Iwahori_\second=G[[t]]$ and $\Iwahori_\zero$ is the standard Iwahori so \begin{equation}\label{eq:A-B}
    \gls{Asph}=\Hom_{\mathscr{B}}(\second,\second)\qquad \EuScript{A}=\Hom_{{\mathscr{B}}}(\zero,\zero)
\end{equation}  Thus, this extended category encodes the structure of $\EuScript{A}$.
\begin{definition}
  Let $\Phi(\acham,\acham')$ be the product of the terms
  $\varphi^+_i-nh$ over pairs $(i,n) \in [1,d]\times \Z$ such that we
  have the inequalities
  \[\varphi_i^{\operatorname{mid}}(\acham)>n \qquad \varphi_i^{\operatorname{mid}}(\acham')<n \] hold.   Let
  $\Phi(\acham,\acham',\acham'')$ be the product of the terms
  $\varphi^+_i-nh$ over pairs $(i,n)\in [1,d]\times \Z$ such that we
  have the inequalities \newseq
  \[\subeqn\label{pmp}\varphi_i^{\operatorname{mid}}(\acham'')>n \qquad
    \varphi_i^{\operatorname{mid}}(\acham')<n \qquad \varphi_i^{\operatorname{mid}}(\acham)>n\] or the inequalities
  \[\subeqn\label{mpm}\varphi_i^{\operatorname{mid}}(\acham'')<n \qquad
    \varphi_i^{\operatorname{mid}}(\acham')>n\qquad \varphi_i^{\operatorname{mid}}(\acham)<n. \] These terms correspond to the hyperplanes that a path
  $\acham\to\acham'\to \acham''$ must cross twice. 
\end{definition}



Recall from \cite[Thm. 3.7]{WebSD} that we have:
\begin{theorem}\label{thm:BFN-pres}
  The morphisms in the extended BFN category are generated by
  \begin{enumerate}
  \item $\gls{yw}_w$ for $w\in \gls{What}$, the graph of a lift of $w$:
\begin{equation*}
\gls{yw}_w=[\big\{(w,v(t))\mid v(t)\in
U_\acham\big\}]/\Iwahori_\acham;\label{eq:y-def}
\end{equation*}
  \item $\glslink{r}{r(\acham,\acham')}$ for $\acham,\acham'\in \ft_{1,\To,\R}$
    generic
\begin{equation*}
r(\acham,\acham')=\left[\{(e,v(t)) \in T((t))\times U_{\acham'}\mid v(t)\in
U_{\acham}\}/T[[t]]\right]\in \Hom_{\ab}(\acham',\acham) ;\label{eq:r-def}
\end{equation*}
  \item $\glslink{Bpsi}{u_{\al'}}(\acham)=\glslink{Bpsi}{u_{\al-n\delta}}(\acham)$ for
    $\acham_\pm$ affine chambers adjacent across $\al'(\acham)=0$ for
    $\al'\in \hatD$ an affine root (i.e. $\al'=\al-n\delta$ for some
    finite root $\al'$)
\begin{equation*}
\glslink{Bpsi}{u_{\al'}}(\acham)=[{\left\{(gv(t),g \cdot\Iwahori_{\pm} ,g\cdot \Iwahori_{\mp})\in \VB_{\acham^\pm}\times_{V((t))}\VB_{\acham^{\mp}}
  \mid g\in
  G((t)), v(t)\in U_\acham\right\}}];\label{eq:psi-def}
\end{equation*}
  \item the polynomials in $\gls{S}_h$.
  \end{enumerate}
  This category has a polynomial representation where each object $\acham$ is assigned to $H_*^{BM,\wtG}(\VB_{\acham})\cong \Cth\cdot [\VB_{\acham}]$, and the generators above act by:
  \newseq \begin{align*}
\subeqn\label{eq:ract} 
\glslink{r}{r(\acham,\acham')} \cdot f [\VB_{\acham'}]&=\Phi(\acham,\acham') f\cdot
                                           [\VB_{\acham}]\\
  \subeqn\label{eq:psiact}
\gls{Bpsi}\cdot f[\VB_{\acham_{\pm}}]&=\partial_{\al}(f)\cdot [\VB_{\acham_{\mp}}]\\
   \subeqn\label{eq:wact} \gls{yw}_w\cdot f[\VB_{\acham}]&=(w\cdot f)[\VB_{w\cdot \acham}]\\
     \subeqn\label{eq:muact}
\mu \cdot f [\VB_{\acham}]&=\mu f\cdot [\VB_{\acham}]
  \end{align*} \newseq 
The relations between these operators are given by:
 \begin{align*}
\subeqn\label{eq:dot-commute}
\mu \cdot  r(\acham,\acham') &= r(\acham,\acham')\cdot \mu   \\
\subeqn\label{eq:weyl1}
  y_{\zeta}\cdot\mu\cdot y_{-\zeta}&=\mu+h \langle \zeta,\mu\rangle \\
\subeqn\label{eq:wall-cross1}
r(\acham,\acham') r(\acham'',\acham''')&=
\delta_{\acham',\acham''}\Phi(\acham,\acham',\acham''')
                                         r(\acham,\acham''')\\
\subeqn\label{eq:coweight2}
y_w\cdot y_{w'}&=y_{ww'}\\
\subeqn\label{eq:conjugate2}
  y_w r(\acham',\acham) y_w^{-1}&=r(w\cdot \acham',w\cdot\acham) \\
\subeqn\label{eq:weyl2}
  y_w \mu y_w^{-1}&=w\cdot \mu\\
   \subeqn\label{eq:psi2}
\Bpsi_{\al}^2&=0\\
   \subeqn\label{eq:psi}
\underbrace{\Bpsi_{\al}\Bpsi_{s_\al \beta}\Bpsi_{s_\al s_{\beta}\al}\cdots}_{m_{\al\be}}
             &=\underbrace{\Bpsi_{\beta}\Bpsi_{s_{\beta}\al}\Bpsi_{s_{\beta}s_\al\beta}\cdots}_{m_{\al\be}}\\
   \subeqn\label{eq:psiconjugate}
y_w\Bpsi_{\al}y_{w^{-1}}&=\Bpsi_{w\cdot \al}\\
  \subeqn\label{eq:psipoly}
\Bpsi_{\al} \mu-(s_{\al}\cdot\mu)\Bpsi_{\al}
             &=r(\eta_{\mp},\eta_{\pm})\partial_{\al}(\mu) \end{align*} whenever these
           morphisms are well-defined
and finally, if $\acham'_\pm$ and $\acham''_\pm$ are two pairs of chambers
opposite across $\al(\acham)=0$ on opposite sides of an intersection
of affine root and flavor hyperplanes as shown below, and
$\acham,\acham''$ differ by a $180^\circ$ rotation around the
corresponding codimension 2 subspace:
 \addtocounter{subeqn}{1}
\begin{multline*}
\subeqn\label{eq:triple} 
r(\acham''',\acham'_-)\Bpsi_{\al} r(\acham'_+,\acham)
-r(\acham''',\acham''_-)\Bpsi_{\al}
r(\acham''_+,\acham)\\=\partial_\al\left(\Phi(\acham'_+,\acham) \cdot s_{\al}\Phi(\acham,\acham'_-)\right)
r(\acham''',s_\al\acham) s_\al.
\end{multline*}
\end{theorem}

One important change in the characteristic $p$ case is that the
representation defined by (\ref{eq:ract}--\ref{eq:muact}) is no longer
faithful, since the same is true of the corresponding representation
of $\widehat{W}$: translations by cocharacters divisible by $p$ act
trivially.

It is possible to fix this, though it is somewhat less pleasant to
think about. Fix $h=g\in \C$ (we will of course be primarily interested in the cases $g=0,1$).  Let $K$ be the fraction field of $\gls{S}_g$, and consider
the induced action by convolution on $K\otimes_{S_g}
H_*^{BM,T}(\VB_{\acham}^T)\cong \oplus_{\la\in X_*(T)}K\cdot
[t^\la]$.  We will not explicitly check that the action we define
below arises from convolution due to fact the complications of
localization in equivariant cohomology for loop groups, but it is
worth pointing to as our source of inspiration.
  \begin{lemma}\label{lem:frac-rep}
There is a faithful action of $\gls{scrB}$  on $K_X:=  \oplus_{\la\in X_*(T)}K\cdot [t^\la]$
given by  the formulas \newseq\begin{align*}
\subeqn\label{eq:ract-frac}
\gls{r}   (\acham,\acham') \cdot f [t^\la]&=\Phi(\acham,\acham') f\cdot
                                           [t^\la]\\
\subeqn\label{eq:psiact-frac}
\gls{Bpsi}\cdot f [t^\la]&=\frac{s_{\al}f}{\al}[t^{s_{\al}\la}]-\frac{f}{\al}[t^{\la}]\\
   \subeqn\label{eq:wact-frac}
\gls{yw}_w\cdot f[t^\la]&=(w\cdot f)[t^{w\la}]\\
     \subeqn\label{eq:muact-frac}
\mu \cdot f [t^\la]&=\mu f\cdot [t^\la].
  \end{align*} 
  \end{lemma}



 
%\subsection{Coherent sheaves}
Now, consider the case where $\hbar=0$.  In this case,
$\End_{\mathscr{B}}(\second,\second)\cong \K[\fM]$ is the space of functions on the
  Coulomb branch.  The different non-isomorphic objects of
  $\gls{scrB}$ define interesting modules over $\K[\fM]$, considering
  $Q_{\eta}=\Hom_{\mathscr{B}}(\zero,\eta)$ as a right module under composition.   
\begin{lemma}\label{lem:Q-rank}
  The module $Q_{\eta}$, considered as a coherent sheaf on $\K[\fM]$, is generically free of rank $\# \gls{W}$.
\end{lemma} 
\begin{proof}
  As a module over $\End_{\mathscr{B}}(\zero,\zero)$ the module $\End_{\mathscr{B}}(\zero,\eta)$ is generically free of rank 1, since $\zero$ and $\eta$ become isomorphic after inverting all weights and roots.  Since at $h=0$, the algebra $\End_{\mathscr{B}}(\zero,\zero)$ is Azumaya over $\K[\fM]$ with degree $\#W$, this implies that $Q_{\eta}$ generically has the correct rank.  
\end{proof}
\excise{
Note that this gives us an isomorphism $Z(\pStein_\K)\cong \K[\fM]$.   It will be an important observation for us later that:
\begin{lemma}\label{lem:center-iso}
 We have an isomorphism $\K[\fM]\to Z(\pStein_\K)$ for $\K=\Z$, and thus for any ring $\K$.
\end{lemma}
\begin{proof}
  The map $\Z[\fM]\to Z(\pStein_\Z)$ is 
  a map between free abelian groups (since both are defined to be subgroups of free abelian groups).  This map is
  an isomorphism when $\K$ is of the form $\mathbb{F}_q$ for any prime $q$, so the map is an isomorphism. 
\end{proof}}

One other construction we'll need to connect to wall-crossing functors
is the twisting bimodules relating two flavors $\gls{flav}$ and
$\flav+\nu$ that differ by $\nu\in \ft_{\Z,\gls{F}}$
  ${}_{\phi+\nu}\gls{scrT}{}_{\phi}$ 
and
${}_{\phi+\nu}\gls{Twist}_{\phi}={}_{\phi+\nu}\mathscr{T}{}_{\phi}(\zero,\zero)$
defined in \cite[Def. 3.16]{WebSD}.
These are constructed much as the Hom spaces in Definition
\ref{def:extended-BFN}: let ${}_{\acham}\VB^{(\nu)}_{\acham'}$ be the
component of the space ${}_{\acham}\gls{dVB}^{\To}_{\acham'}$  as defined in
\eqref{eq:aXa} lying above $t^\nu$ in the affine Grassmannian of
$t^\nu$, and we have:
\begin{equation}
  \label{eq:aXnua}
 \glslink{scrT}{ {}_{\phi+\nu}\mathscr{T}{}_{\phi}}(\acham,\acham')=  H_*^{BM, \tilde T}\left({}_{\acham}\VB^{(\nu)}_{\acham'}; 
    \K\right).
\end{equation}


\subsection{Representations}
\label{sec:reps}
Using this presentation, we can analyze the structure of this category of representations in
characteristic $p$, just as we did in characteristic 0 in \cite[\S
3.3]{WebSD}.   It is worth noting that the group $\gls{G}$, representation
$\gls{V}$ and its associated objects are unchanged; we simply consider their
homology over $\K$, a field of characteristic $p$.  To save ourselves
heart-burn, we assume that $p$ is not a torsion prime for the group
$G$.  This is not a problematic restriction, since we will typically
assume that $p\gg 0$.    Throughout this subsection, we specialize $h=1$.


Let $M$ be a finite dimensional representation
of the category $\gls{scrB}$ (which we will also call $\mathscr{B}$-modules), that is, a functor from $\mathscr{B}$ to the
category $\K\Vect$ of finite dimensional $\K$-vector spaces.  

These are closely related to the theory of $\EuScript{A}$-modules since the finite dimensional vector space $N:=M(\zero)$ has an
induced $\EuScript{A}$-module structure.  Furthermore, since $\Hom(\acham,\zero)$
and $\Hom(\zero,\acham)$ are finitely generated as
$\EuScript{A}$-modules, this is in fact a quotient functor, with left adjoint given by 
\[N\mapsto
\mathscr{B}\otimes_{\EuScript{A}}N(\acham):=\Hom(\acham,\zero)\otimes_{\EuScript{A}}N.\]

Now, let us return to the theory of $\mathscr{B}$-modules Of course, if we restrict the action on $M(\eta)$ to the subalgebra
$\gls{S}_h$, then this vector space breaks up as a sum of {\bf weight spaces}:
\begin{equation}
  \label{eq:W-def}
  \Wei_{\upsilon,\acham}(M)=\{m\in M(\acham)\mid \mathfrak{m}_{\upsilon}^Nm=0 \text{ for } N\gg 0\},
\end{equation} 
Of course, we can think of this as an exact functor
$\Wei_{\upsilon,\acham}\colon \mathscr{B}\operatorname{-fdmod}\to \K\Vect$.
In \cite{WebSD}, we employed these weight functors to probe the
category of representations of  $\mathscr{B}$.   Versions of this
construction have appeared a number of places in the literature,
including work of Musson and van der Bergh \cite{MVdB} and Drozd,
Futorny and Ovsienko \cite{FOD}.
\begin{definition}
  Let $\gls{scrBhat}$ be the category whose objects are the
  set $\EuScript{J}$ of pairs of generic $\acham\in \ft_\R+\tau$ and any
  $\upsilon\in \ft_{1,\K}$, such that
  \[\Hom_{\widehat{\mathscr{B}}}((\acham',\upsilon'),(\acham,\upsilon))=\varprojlim
  \Hom_{{\mathscr{B}}}(\acham',\acham)/(\mathfrak{m}_{\upsilon}^N
  \Hom_{{\mathscr{B}}}(\acham',\acham)+\Hom_{{\mathscr{B}}}(\acham',\acham)\mathfrak{m}_{\upsilon'}^N).\] 
\end{definition}
We can apply \cite[Theorem B]{WebGT} here to get a sense of the size
of this algebra: the endomorphism algebra
of any object in this category is again a Galois order in a skew group
algebra, but the group is now the stabilizer of $\acham'$ in the
affine Weyl group. Since we are now in characteristic $p$, this
contains all translations that are $p$-divisible, and so this
stabilizer is the semi-direct product of a parabolic subgroup in the
finite Weyl group with this $p$-scaled group of translations.  


Note that since $\K$ is of characteristic $p$, the set $\ft_{1,\K}$
has an action of $\ft_{\Z/p\Z}$ by addition.
We let $\glslink{scrBhatup}{\mathscr{\widehat{B}}_{\upsilon'}}$ be the subcategory where we only
allow objects  with $\upsilon\in \upsilon'+\ft_{\Z/p\Z}$ and let
$\gls{scrAhatup}_{\upsilon'}$ be the subcategory with the objects of the form
$(\zero,\upsilon)$ for $\upsilon \in \upsilon'+\ft_{\Z/p\Z}$.  

This category is useful in that it lets us organize how the weight spaces
of different values of $\acham$ relate.  For any $\mathscr{B}$-module
$M$, the functor $(\upsilon, \acham)\mapsto W_{\upsilon,\acham}(M)$
defines a representation of $\widehat{\mathscr{B}}$.  We have an
analogue in this situation of \cite[Lemma 3.11]{WebSD}, which is a
special case of a more general result of Drozd-Futorny-Ovsienko
\cite[Th. 17]{FOD}:
\begin{lemma}
  The functor above defines an equivalence of the category $\gls{scrB}\mmod_{\upsilon'}$ of finite dimensional $\gls{scrB}$-modules with weights in $\gls{What}\cdot \upsilon'$ to
  the category of representations of
  $\widehat{\mathscr{B}}_{\upsilon'}$ in $\K\Vect$.

The analogous functor defines an equivalence of the category $\EuScript{A}\mmod_{\upsilon'}$ of finite dimensional
$\EuScript{A}$-modules  with weights in
$\widehat{W}\cdot \upsilon'$ to the category of
representations of $\gls{scrAhatup}_{\upsilon'}$ in $\K\Vect$.
\end{lemma} 
Note that an important difference
between the characteristic $p$ and characteristic $0$ cases:
a module in $\EuScript{A}\mmod_{\upsilon'}$ with $\K$ of characteristic $p$ with finite dimensional
weight spaces is necessarily finite dimensional (since the affine Weyl
group orbit of any weight is finite) while it is typically not if $\K$
has 
characteristic $0$. This is why here we only study finite dimensional
modules, while in \cite{WebSD}, we study the category of all weight modules.


\subsection{The homogeneous presentation}
\label{sec:homo}
We wish to give a homogeneous presentation of the categories
$\gls{scrAhatup}_{\upsilon'}$ and
$\glslink{scrBhatup}{\mathscr{\widehat{B}}_{\upsilon'}}$, as in
\cite[\S 4]{WebSD}.  We assume that $\K$ is a field of characteristic
$p$ not dividing $\# \gls{W}$ and we are still specializing $h=1$.  
% The
% corresponding category in characteristic $p$ is defined relatively
% similarly to the Steinberg category in \cite[\S 2]{WebSD}, but unlike
% that situation, it does not have an obvious geometric interpretation.


Recall that we have fixed $\gls{\phi}\in \ft_{F,\Z}$; we can without
loss of generality choose a lift $\tilde{\phi}$ which is fixed by the action of $W$  to $\ft_{1,\To,
  \Z[\frac{1}{\# W}]}$ over the ring of integers with the order of the
group $W$ inverted 
(note that this might not be
possible over $\Z$).   This has a unique reduction to
$\gls{ft1}_{1,\Fp}$, which is again $W$-invariant.

For fixed $\upsilon'\in \gls{ft1}_{1,\K}$, let $\gls{relweights}_{\upsilon'}\subset
[1,n]$ be the indices such that $\varphi_i(\upsilon')\in \Z/p\Z$.
Similarly, let $\gls{relroots}_{\upsilon'}$ be the set of roots such that
$\al_i(\upsilon') \in \Z/p\Z$. 
In \cite[\S 4]{WebSD}, we call these weights and roots   {\bf
  relevant}.  Note that if $\K=\Fp$, then all roots and weights will
be relevant.   

By averaging over the subgroup $W_{\upsilon'}$ generated by $s_{\al_i}$ for $i\in \weights_{\upsilon'}$, we may assume that for $i
\in  \weights_{\upsilon'}$, we have $\varphi_i(\upsilon')=0$. Thus, the
stabilizer $\widehat{W}_{\upsilon'}$ of $\upsilon'$ in the affine Weyl
group is generated by $s_{\al_i}$ for $i\in \weights_{\upsilon'}$, and 
translations by $p \ft_{\Z}$.  Let $\widehat{W}_{ \weights}$
be the subgroup generated by $s_{\al_i}, i\in \weights_{\upsilon'}$, and 
translations by $\ft_{\Z}$.  Note that the map \[{(\cdot )}_p\colon \widehat{W}_{ \weights} \to \widehat{W}_{\upsilon'} \qquad w_p(x)= p\cdot w\Big(\frac{1}{p}x\Big)\]
is an isomorphism between these groups.  Furthermore, for reflection
in an affine root $\al$, we have that $(s_\al)_p=s_{\al^{(p)}}$ for
some possibly different root $\al^{(p)}$.  

We can also understand this homomorphism in terms of the Frobenius map $\frob_V\colon V((t))\to V((t))$ given by $\frob_V(v(t))=v(t^p)$: the element $w_p$ is the unique one satisfying $w_p\circ \frob_V=\frob_V \circ w$.  Similarly, we will want to understand the interaction between $U_{\eta}$ and this map.  Consider the $\tilde{G}\times \C^*$ action on $V((t))$ as usual (that is, with $\C^*$ acting by loop rotation).  The map $\frob_V$ intertwines the action of 
$G((t))$ with the action twisted by the endomorphism $\frob_G(g(t))=g(t^p)$.

Note, we cannot extend this automorphism to the semi-direct product incorporating the loop scaling, since we would need to act on the loop $\C^*$ by $s\mapsto s^{1/p}$.  The corresponding automorphism on the level of Lie algebras is well-defined however, and we denote it by $\frob_{\tilde{\fg}}$.   Note that this does not preserve the subalgebra $\{(X,d\nu(X))\mid X\in \tilde{\fg}\}$, and thus does not preserve $\widetilde{\fg((t))}$. 

This shows that we need to have a different flavor in order to write $\frob_V^{-1}(U_\eta)$ as the same sort of subspace.  
\begin{definition}\label{def:pth-root}
  The {\bf ``\gls{pthroot}''} conventions for the extended category are taking:
  \begin{itemize}
    \item the gauge group $G_{\upsilon'}$ and representation $V_{\upsilon'}$ corresponding to only the relevant roots and weights;
      \item the (rational) flavor $\phi_{1/p}(t)=(\phi_0(t^{1/p}),t)$ in place of $\gls{flav}$;
      \item the constant $\delta/p$ in place of $\delta$;
      \item specialize the equivariant parameter $h=0$.
  \end{itemize}
  Throughout, we will use sans-serif letters to denote objects defined
  in the \gls{pthroot} conventions. In particular, we write
  $\mathsf{t}_{1,\R}$ for  $\gls{ft1}_{1,\R}$,  $\mathsf{U}_\eta$ for
  $U_{\eta}$.  We let $\gls{sfB}$ for $\gls{scrB}$ when we use this
  modified flavor and constant. 
\end{definition}

\begin{definition}
  Given $\eta\in \mathsf{t}_{1,\To,\R} $, let $\eta_p\in\gls{ft1}_{1,\To,\R}$ be the
  element  $\eta_p=p\cdot \frob_{\mathfrak{\tilde{g}}}(\eta)$.  Given
  $\xi\in \ft_{1,\To,\R}$, let  $\xi_{1/p}\in \mathsf{t}_{1,\To,\R}$ be the unique solution to $\xi=(\xi_{1/p})_p$, that is, the inverse map.
\end{definition}
Note that for $w\in \widehat{W}$, we have $(w\eta)_p=w_p\eta_p$, and that $\gls{varphimid}^{\operatorname{mid}}_i(\eta_p)=p\mathsf{\varphi}^{\operatorname{mid}}_i(\eta)$ (where the second is calculated using $p$th root conventions).
\begin{lemma}
  $\frob_V^{-1}(U_{\eta_p})=\mathsf{U}_{\eta}$.
\end{lemma}
\begin{remark}
  The map $\xi\mapsto \xi_{1/p}$ has the effect of shrinking the space
  $\ft_{1,\To,\R}$ by a factor of $\frac{1}{p}$, and the unrolled matter
  hyperplanes, which are defined by  $\gls{varphimid}_i^{\operatorname{mid}}(\xi)$ taking an
  integral value, become the hyperplanes where this same function
  (with the $p$th root conventions)  takes on a value in
  $\frac{1}{p}\Z$.  Thus we only keep every $p$th one of these
  hyperplanes as a matter hyperplane.

  Each unrolled matter hyperplane separated the locus of $\eta$ such
  that a given weight vector in $V((t))$ lies in $U_\eta$ from the
  locus where is does not; the hyperplanes we keep in the $p$th root
  conventions are those that correspond to vectors in $V((t^p))$.
\end{remark}

\begin{example}
  Let us consider the running example from \cite{WebSD}: the gauge
  group $\gls{G}=GL(2)$ acting on $\gls{V}=\C^2\oplus \C^2$.  The flavor group $\gls{F}$
  is isomorphic to $PGL(2)$, so choosing a flavor is choosing a
  cocharacter into this group, fixing the difference between the
  weights of this cocharacter on the two copies of $\C^2$.  Let's
  consider $p=5$, and choose the flavor so that the difference is
  $\varphi_1^{\operatorname{mid}}-\varphi_3^{\operatorname{mid}}=3$.
  In Figure \ref{fig:pthroot}, we draw the images under $\acham\mapsto
  \acham_{1/p}$ of all unrolled matter hyperplanes, but draw those
  which do not remain as matter hyperplanes for the $p$th root data in
  gray and with thinner weight.
  \begin{figure}\label{fig:pthroot}
    \centering
    \[\tikz[very thick,scale=1.4]{ \draw[old]
    (1.16,2.5)-- (1.16,-3.5) node[scale=.5, at
    end,below]{$\varphi_3^{\operatorname{mid}}=-1/5$}  node[scale=.5, at
    start,above]{$\varphi_1^{\operatorname{mid}}=2/5$}; % \draw (-1.16,2.5)-- (-1.16,-3.5)
    % node[scale=.5, at
    % end,below]{$\varphi_3=-2/3$}; 
    \draw[old] (2.5,1.16)-- (-3.5,1.16)
    node[scale=.5, at
    end,left]{$\varphi_4^{\operatorname{mid}}=-1/5$}
    node[scale=.5, at
    start,right]{$\varphi_2^{\operatorname{mid}}=2/5$};
    %\draw (2.5,-1.16)-- (-3.5,-1.16) node[scale=.5, at
    %end,left]{$\varphi_4=-2/3$};
    \draw (2.16,2.5)-- (2.16,-3.5)
    node[scale=.5, at
    end,below]{$\varphi_3^{\operatorname{mid}}=0$}
    node[scale=.5, at
    start,above]{$\varphi_1^{\operatorname{mid}}=3/5$}; % \draw (-.16,2.5)-- (-.16,-3.5)
    % node[scale=.5, at
    % end,below]{$\varphi_3=1/3$}; 
    \draw (2.5,2.16)-- (-3.5,2.16)
    node[scale=.5, at
    end,left]{$\varphi_4^{\operatorname{mid}}=0$}
    node[scale=.5, at
    start,right]{$\varphi_2^{\operatorname{mid}}=3/5$};
    %\draw (2.5,-.16)-- (-3.5,-.16)
    %node[scale=.5, at
    %end,left]{$\varphi_4=1/3$};
    \draw[old] (0.16,2.5)-- (0.16,-3.5)
    node[scale=.5, at
    end,below]{$\varphi_3^{\operatorname{mid}}=-2/5$} 
    node[scale=.5, at
    start,above]{$\varphi_1^{\operatorname{mid}}=1/5$}; % \draw (.84,2.5)-- (.84,-3.5)
    % node[scale=.5, at
    % end,below]{$\varphi_3=4/3$}; 
    \draw[old] (2.5,.16)-- (-3.5,.16) 
    node[scale=.5, at
    end,left]{$\varphi_4^{\operatorname{mid}}=-2/5$}
    node[scale=.5, at
    start,right]{$\varphi_2^{\operatorname{mid}}=1/5$}; % \draw (2.5,.84)-- (-3.5,.84)
    % node[scale=.5, at
    % end,left]{$\varphi_4=4/3$}; 
    \draw (-.84,2.5)-- (-.84,-3.5)
    node[gray,scale=.5, at
    end,below]{$\varphi_3^{\operatorname{mid}}=-3/5$}
    node[scale=.5, at
    start,above]{$\varphi_1^{\operatorname{mid}}=0$}; % \draw (1.84,2.5)-- (1.84,-3.5)
    % node[scale=.5, at
    % end,below]{$\varphi_3=7/3$};
    \draw (2.5,-.84)-- (-3.5,-.84) 
    node[gray,scale=.5, at
    end,left]{$\varphi_4^{\operatorname{mid}}=-3/5$}
    node[scale=.5, at
    start,right]{$\varphi_2^{\operatorname{mid}}=0$}; % \draw (2.5,1.84)-- (-3.5,1.84)
    % node[scale=.5, at
    % end,left]{$\varphi_4=7/3$}; 
    \draw[old] (-1.84,2.5)-- (-1.84,-3.5) node[scale=.5, at
    end,below]{$\varphi_3^{\operatorname{mid}}=-4/5$} 
    node[scale=.5, at
    start,above]{$\varphi_1^{\operatorname{mid}}=-1/5$}; % \draw (-2.16,2.5)-- (-2.16,-3.5)
    % node[scale=.5, at
    % end,below]{$\varphi_3=-5/3$}; 
    \draw[old] (2.5,-1.84)-- (-3.5,-1.84) 
    node[scale=.5, at
    end,left]{$\varphi_4^{\operatorname{mid}}=-4/5$}
    node[scale=.5, at
    start,right]{$\varphi_2^{\operatorname{mid}}=-1/5$}; % \draw
    % (2.5,-2.16)-- (-3.5,-2.16)
    
    \draw (-2.84,2.5)-- (-2.84,-3.5) node[scale=.5, at
    end,below]{$\varphi_3^{\operatorname{mid}}=-1$} 
    node[gray, scale=.5, at
    start,above]{$\varphi_1^{\operatorname{mid}}=-2/5$}; % \draw (-2.16,2.5)-- (-2.16,-3.5)
    % node[scale=.5, at
    % end,below]{$\varphi_3=-5/3$}; 
    \draw (2.5,-2.84)-- (-3.5,-2.84) 
    node[scale=.5, at
    end,left]{$\varphi_4^{\operatorname{mid}}=-1$}
    node[gray, scale=.5, at
    start,right]{$\varphi_2^{\operatorname{mid}}=-2/5$}; % \draw (2.5,-2.16)-- (-3.5,-2.16)
    % node[scale=.5, at
    % end,left]{$\varphi_4=-5/3$}; 
    \draw[dotted] (-3.5,-3.5) -- node[scale=.5, right,at
    end]{$\alpha=0$}(2.5,2.5); \draw[dotted,gray, thin] (-3.5,-2.5) --
    node[scale=.5, below left,at
    start]{$\alpha=1/5$}(1.5,2.5); \draw[dotted,gray, thin]
    (-2.5,-3.5) -- node[scale=.5, above right,at
    end]{$\alpha=-1/5$}(2.5,1.5); \draw[dotted,gray, thin] (-3.5,-1.5)
    -- node[scale=.5, below left,at
    start]{$\alpha=2/5$}(0.5,2.5); \draw[dotted,gray, thin]
    (-1.5,-3.5) -- node[scale=.5, above right,at
    end]{$\alpha=-2/5$}(2.5,0.5); \draw[dotted,gray, thin] (-3.5,-.5)
    -- node[scale=.5, below left,at
    start]{$\alpha=3/5$}(-.5,2.5); \draw[dotted,gray, thin] (-.5,-3.5)
    -- node[scale=.5, above right,at
    end]{$\alpha=-3/5$}(2.5,-.5); \draw[dotted,gray, thin] (-3.5,.5)
    -- node[scale=.5, below left,at
    start]{$\alpha=4/5$}(-1.5,2.5); \draw[dotted,gray, thin] (.5,-3.5)
    -- node[scale=.5, above right,at
    end]{$\alpha=-5$}(2.5,-1.5); \draw[dotted] (-3.5,1.5) --
    node[scale=.5, below left,at
    start]{$\alpha=1$}(-2.5,2.5); \draw[dotted] (1.5,-3.5) --
    node[scale=.5, above right,at end]{$\alpha=-1$}(2.5,-2.5);}\]
\caption{The effect of $p$th root conventions on matter hyperplanes}
\end{figure}
\end{example}


Let $V_{\upsilon'}$ be the span of the vectors $v_i$
for $i\in\gls{relweights}_{\upsilon'}$, and $L$ the Levi subgroup for
$\gls{relroots}_{\upsilon'}$.  Note that $V_{\upsilon'}$ defines a
representation of $L$.  
We'll want to consider the affine chambers for $\Ba\in \Z^{\weights_{\upsilon'}}$ as in Section \ref{eq:aff-cham}: 
\[\gls{rACp}_{\Ba}=\{\xi \in \mathsf{t}_{1,\R}\mid a_i<\gls{varphimid}_i^{\operatorname{mid}}(\xi)<a_i+1\text{
  for all $i\in \gls{relweights}_{\upsilon'}$}\}\label{eq:r-aff-cham}\]  
\[\ACs=\{\Ba\in \Z^{\weights_{\upsilon'}}\mid \rACp_{\Ba}\neq 0\}.\] 
  The reader should note that we use the \gls{pthroot} conventions here.
Consider $\widehat{W}$, the extended affine Weyl group, acting on
$\mathsf{t}_{1,\To,\R}$ via the usual level 1 action. Note that if $w\in
\widehat{W}$ and $\rACp_{\Ba}\neq 0$ then $w\cdot
\rACp_{\Ba}=\rACp_{w\cdot \Ba}$ for a unique $w\cdot \Ba$, so this
defines a $\widehat{W}$ action on $\ACs$.

First, we note how the polynomial changes when we complete it to match
$\glslink{scrBhatup}{\widehat{\mathscr{B}}_{\upsilon'}}$.  
Given $\upsilon$, we
let $\widehat{S}^{(\upsilon)}_1=\varprojlim
\glslink{S}{S_1}/\mathfrak{m}_\upsilon^N$.

\begin{proposition}\label{prop:hat-rep}
  The category $\glslink{scrBhatup}{\mathscr{\widehat{B}}_{\upsilon'}}$ has a 
  representation $\mathscr{P}$ sending $(\acham, \upsilon) \mapsto
  \widehat{S}^{(\upsilon)}_1$ defined by the formulas (\ref{eq:ract}--\ref{eq:muact}), and one $\mathscr{F}$ sending 
  \[(\acham, \upsilon)\mapsto \bigoplus \operatorname{Frac}( \widehat{S}^{(\upsilon)}_1)\cdot [t^\la] \]
  where $\operatorname{Frac}(-)$ denotes the fraction field of a commutative ring, with morphisms acting as in (\ref{eq:ract-frac}--\ref{eq:muact-frac}).
\end{proposition}
Since $\widehat{S}^{(\upsilon)}_1$ is a profinite dimensional algebra,
we can decompose any element of it into its semi-simple and nilpotent
parts, by doing so in each quotient $S_1/\mathfrak{m}_\upsilon^N$.
The grading we seek on a dense subcategory of  $\widehat{\mathscr{B}}_{\upsilon'}$ is uniquely
fixed a small list of  requirements:
\begin{enumerate}[label=(\roman*)]
\item For $\mu \in \ft^*$, the nilpotent part $\mu-\langle\mu,\upsilon\rangle$ of $\mu$
  acting on $\widehat{S}^{(\upsilon)}_1$ is homogeneous of degree 2.
\item The action of $\widehat{W}$ is homogeneous of degree 0.
\item If $\acham\in \rACp_{\Ba}$ and $\acham'\in \rACp_{\Bb}$ then the
  obvious isomorphism   $\mathscr{P}(\acham_p,\upsilon)\cong \widehat{S}^{(\upsilon)}_1\cong \mathscr{P}(\acham'_p,\upsilon)$
is homogeneous of degree $\sum_{i\in \weights_\upsilon'} a_i-b_i$.
\end{enumerate}
The principles (i-iii) fix a grading on $\mathscr{P}(\acham,\upsilon)$
for all $\acham\in \ft_{1,\To,\R}$ and $\upsilon\in \hat{W}\cdot
\upsilon'$ up to a global shift. 

We would like to use this to fix a notion of what it means
for a morphism in $\widehat{\mathscr{B}}_{\upsilon'}$ to be
homogeneous, however this is slightly complicated by the fact that
$\mathscr{P}$ is not faithful.  However, it can still be a useful guide
to the choice of an appropriate grading.  

Let $\hat \Phi_0(\acham,\acham',\upsilon')$ be the product of the terms
$\varphi^+_i-n$  over pairs $(i,n) \in [1,d]\times\Z$ such that we have the inequalities
\[\glslink{varphimid}{\varphi_i^{\operatorname{mid}}}(\acham)>n\qquad \varphi_i^{\operatorname{mid}}(\acham')<n \qquad
\langle\varphi^+_i,\upsilon'\rangle\not\equiv n\pmod p.\]  Note that these are precisely the factors in the product $\Phi(\acham,\acham')$ which remain invertible after reduction modulo $p$.  
Consider the morphisms:
  \begin{equation*}
  \wall(\acham,\acham')=
                            \frac{1}{\hat{\Phi}_0(\acham,\acham',\upsilon')}r(\acham,\acham)
 \end{equation*}
We'll check below that these morphisms are homogeneous, and together
with a few other obvious homogeneous morphisms, they generate a dense
subspace inside morphisms.






Consider the extended BFN category $\gls{sfB}$ with \gls{pthroot} conventions.  Since $h=0$, this category is graded with   
\begin{samepage}
    \newseq\begin{equation*}\subeqn
    \label{eq:Stein-grading1}
\deg \glslink{r}{r(\acham,\acham')}=\deg \Phi(\eta,\eta')+\deg \Phi(\eta',\eta)
\end{equation*}\begin{equation*}\subeqn
\deg w=0\qquad \deg u_\al(\Ba)=-2 \qquad  \deg
  \mu=2\label{eq:Stein-grading2}.
\end{equation*}
\end{samepage}
Note that here $\deg \Phi(\eta,\eta')$ should be interpreted in the grading on $\glslink{S}{S_0}$ where $\ft^*$ is concentrated in degree 1.  
We define a functor $\gamma_{\mathsf{B}}\colon \gls{sfB}\to
\glslink{scrBhatup}{\mathscr{\widehat{B}}_{\upsilon'}}$ by sending
$\eta\mapsto (\eta_p-\tilde{\phi},
\upsilon')$, and acting on morphisms by 
\newseq 
  \begin{align*}
    \gamma_{\mathsf{B}}(\glslink{r}{r(\acham,\acham')})&=\wall(\eta_p-\tilde{\phi},\eta_p'-\tilde{\phi})= \frac{1}{\hat{\Phi}_0(\eta_p-\tilde{\phi},\eta_p'-\tilde{\phi},\upsilon')}r(\eta_p-\tilde{\phi},\eta_p'-\tilde{\phi})\subeqn \label{gamma1}\\
    \gamma_{\mathsf{B}} (w)&= w_p \subeqn \label{gamma2}\\
    \gamma_{\mathsf{B}}(u_\al) &=u_{\al^{(p)}}\subeqn \label{gamma3}\\
    \gamma_{\mathsf{B}}(\mu)&= \mu-\langle\mu,\upsilon'\rangle \subeqn \label{gamma4}
  \end{align*}
Note that since $w_p\acham_p=(w\acham)_p$ and $w_p\cdot \upsilon'=\upsilon'$, these morphisms go
between the correct objects.

\begin{remark}
  Just as discussed in \cite{WebSD}, this isomorphism has a natural
  geometric interpretation as localization to the fixed points of a
  group action.  In the characteristic zero case, we analyze the space
  corresponding to a weight in terms of the fixed points of the
  corresponding character; a version of this is explained in
  \cite{WebGT}, generalizing work of Varagnolo-Vasserot \cite[\S
  2]{MR3013034}. In characteristic $p$, we only obtain an isomorphism
  after completion with the fixed points of the $p$-torsion subgroup
  of this cocharacter.  Of course, these fixed points again give
  versions of the spaces appearing in the extended BFN category.
  While this is a beautiful perspective, we think it will be clearer,
  especially for the reader unused to the geometry of the affine
  Grassmannian, to give an algebraic proof.
\end{remark}

  \begin{proposition}\label{prop:B-equiv}
    The functor $\gamma_{\mathsf{B}}\colon \gls{sfB}\to \glslink{scrBhatup}{\mathscr{\widehat{B}}_{\upsilon'}}$ is  faithful, topologically full and essentially surjective; that is, it induces an equivalence $\gls{sfBhat}\cong \glslink{scrBhatup}{\widehat{\mathscr{B}}_{\upsilon'}}$
  \end{proposition}
  
  
  \begin{proof}
  Consider the representation $\mathscr{F}$ of  $\widehat{\mathscr{B}}_{\upsilon'}$.  In order to confirm the result, we must show that the images above under $\gamma_{\mathsf{B}}$ satisfy the relations of $\gls{sfB}$ and define a faithful representation on $\mathscr{F}$. 
  
  The formula \ref{gamma4} identifies each summand of $\mathscr{F}$ with the fraction field of the completion of $\widehat{S}_0^{(0)}$ at the origin of $S_0$, and thus $\mathscr{F}(\eta_p,\upsilon')\cong \bigoplus \widehat{S}_0^{(0)}\cdot [t^\la]$.  Consider the images of the RHSs of (\ref{gamma1}--\ref{gamma4}) under transport of structure.  We wish to show that this agrees with the representation $\mathsf{K}$ of $\mathsf{B}$ defined in Lemma \ref{lem:frac-rep}.  
  
For the morphism $\gamma_{\mathsf{B}}(\mu)$, this is automatic.  For $\gamma_{\mathsf{B}}(w)$, this is an immediate consequence of the definition.  
  
Now, consider $r(\eta,\eta')$.  In its polynomial
representation, this element acts by
${\mathsf{\Phi}(\acham,\acham')}$, which is the product over $i\in
\gls{relweights}_{\upsilon'}$ of $\varphi_i^+$ raised to number of
integers $n$ satisfying ${\varphi}^{\operatorname{mid}}_i(\acham)< n<
{\varphi}^{\operatorname{mid}}_i(\acham')$.  This is sent under
$\gamma_{\mathsf{B}}$ to the shift $\varphi_i^+-\langle
\varphi^+,\upsilon'\rangle$, leaving the overall structure of the
product the same.  

Now note that for $i\in \gls{relweights}_{\upsilon'}$, we have ${\varphi}^{\operatorname{mid}}_i(\acham)< n< {\varphi}^{\operatorname{mid}}_i(\acham')$ if and only if $\varphi^{\operatorname{mid}}_i(\acham_p)< pn< \varphi^{\operatorname{mid}}_i(\acham'_p)$.  
Thus, we have an equality  \[{\mathsf{\Phi}(\acham,\acham')}=\frac{{\Phi}(\acham_p,\acham'_p)}{\hat{\Phi}_0(\acham_p,\acham_p,\upsilon')}.\]
Note the difference between $\mathsf{\Phi}$ and $\Phi$ in the equation above, denoting the use of \gls{pthroot} conventions on the left hand side.  Thus, indeed, in $\bigoplus \widehat{S}_0^{(0)}$, as desired we have that $\gamma_{\mathsf{B}}(r(\eta,\eta'))=\wall(\eta_p,\eta_p')$ acts by multiplication by $\mathsf{\Phi}(\acham,\acham')$.  Finally, $\gamma_{\mathsf{B}}(u_\al)$ must have the desired image because it can be written as $\frac{1}{\al}(s_{\al}-r(s_\al\acham,\acham))$ when it is well-defined and $\al$ is invertible in $\operatorname{Frac}(\widehat{S}_0^{(0)}).$  This shows that we have recovered $\mathsf{F}$.

Since $\mathsf{F}$ is a faithful representation, this shows that $\gamma_{\mathsf{B}}$ is well-defined and faithful.

Any object $(\acham, w\cdot \upsilon')$ is in the essential image of
the functor, since it was isomorphic to $(w^{-1}\cdot \acham,
\upsilon')$, and the map $(\cdot)_p$ is a bijection. 

Finally, we need to show that the image of the functor is dense.
Note that:   \begin{align*}
  \glslink{r}{r(\acham,\acham')}  &=\hat{\Phi}_0(\acham,\acham',\upsilon')\wall(\acham,\acham')=
                            \hat{\Phi}_0(\acham,\acham',\upsilon')\gamma_{\mathsf{B}}(r(\acham_{1/p},\acham'_{1/p}))\\
   w&=  \gamma_{\mathsf{B}} (w_{1/p})\\
   u_{\al^{(p)}}&=\gamma_{\mathsf{B}}(u_\al) \\
   \mu &=  \gamma_{\mathsf{B}}(\mu+\langle\mu,\upsilon'\rangle). 
  \end{align*}
We are only left the task of showing that $u_{\al}$ is
in the closure of the image of $\gamma_{\mathsf{B}}$ for $s_{\al}\notin \widehat{W}_{\upsilon'}$.  In this case, $\al$
thought of as an element of $S_1$ will act invertibly, so $1/\al$ lies in the closure of the image.   Thus, we can use the
formula $\frac{1}{\al}(s_{\al}-r(s_\al\acham,\acham))$ as before. This shows the density.
  \end{proof}


\begin{remark}\label{rem:coefficients}
  We should emphasize that the functor $\gamma_{\mathsf{B}}$ is only well-defined over a field of characteristic $p$, but the category $\gls{sfB}$  makes sense with coefficients in  an arbitrary commutative ring (in
  particular, $\Z$).  We'll write $\gls{sfB}(\K)$ when we wish to emphasize the choice of base field.  
\end{remark}


\begin{definition}\label{def:sfA}
  We let $\gls{sfA}(\K)$ be the category whose objects are 
 the elements of $\upsilon'+\gls{ft1}_{1,\Z}$, with morphisms 
 $\Hom_{ \mathsf{A}_p(\K)}(\xi,\xi')\cong
 \Hom_{\mathsf{B}(\K)}(-\xi_{1/p}, -\xi_{1/p}')$, and
 $\widehat{\mathsf{A}}_p (\K)$ its completion with respect to the grading. 
\end{definition}
 Note that we have broken a bit from our convention of using sans-serifizing (which would suggest that $\mathsf{A}$ should be the ring $\EuScript{A}_0$); the reason for this will be clearer below.

Since objects in $\mathsf{A}_p(\K)$ that differ by the level $p$ action of $\gls{What}$ are
isomorphic, we could also consider only the elements  of
$\upsilon'+\ft_{1,\Z}$ in a fundamental region for the action of this group.  However, it's more convenient to describe the elements $\wall$ in the unrolled picture.  

%Now assume that  $\K$ is an Artinian local ring with residue field of
%characteristic $p$ ($\Z/p^N\Z$ being the most important example) and
%that $n=p$.  Let $\fm_\upsilon$ denote the maximal ideal of $S_1$
%corresponding $\upsilon\in \ft_{1,\Z}$, reduced to $\K$.    
%Let $\delta\colon \Z\to \Z$ be the map sending $m\mapsto m-n\lceil
%m/n-\nicefrac 12\rceil$; that is, $\delta(m)$ is the unique integer in
%$[\frac{n}{2},-\frac{n}{2})$. 



Note that in $\glslink{scrBhatup}{\widehat{\mathscr{B}}_{\upsilon'}}$,  the translation by
$\upsilon'-\upsilon $ induces an isomorphism
$(\zero ,\upsilon)\cong (\upsilon'-\upsilon+\zero, \upsilon')$.  Since
$\varphi_i(\upsilon'+\zero)$ is arbitrarily small for $i\in \gls{relweights}_{\upsilon'}$, the element $r(\upsilon'-\upsilon+\zero,\zero)$ induces an isomorphism $(\zero ,\upsilon)\cong (-\upsilon, \upsilon')$

Conjugating the functor $\gamma_{\mathsf{B}}$ by this isomorphism induces a functor $\gamma\colon
\widehat{\mathsf{A}}_p \to\widehat{\mathscr{A}}_{\upsilon'} $ sending $\gamma(\zero,\upsilon)=\upsilon$.  Proposition \ref{prop:B-equiv} immediately implies that:
\begin{theorem}\label{thm:pStein-equiv}
  The categories $\gls{scrAhatup}_{\upsilon'}$ and
  $\widehat{\mathsf{A}}_p$ are equivalent via the functors
 \[\tikz[->,thick]{
\matrix[row sep=12mm,column sep=25mm,ampersand replacement=\&]{
\node (d) {$\widehat{\mathsf{A}}_p$}; \& \node (e)
{$\gls{sfBhat}$}; \\
\node (a) {$\widehat{\mathscr{A}}_{\upsilon'}$}; \& \node (b)
{$\widehat{\mathscr{B}}_{\upsilon'}$}; \\
};
\draw (a) -- (b) ; 
\draw (d) -- (a) node[left,midway]{$\gamma$} ; 
\draw (e) -- (b) node[right,midway]{$\gamma_{\mathsf{B}}$}; 
\draw (d) -- (e) node[above,midway]{$\upsilon\mapsto -\upsilon_{1/p}$}; 
}\]
\end{theorem}



\subsection{Consequences for representation theory}
\label{sec:cons-repr-theory}
Theorem \ref{thm:pStein-equiv} on its own has a quite interesting
consequence for  the behavior of the finite dimensional
representations of $\glslink{efA}{\EuScript{A}_\phi}$ for
different primes $p$ with  $\gls{flav}/p$  ``held constant.''  For simplicity in this section, we only consider the case where $\K=\mathbb{F}_p$, though the results could be
generalized without must difficulty.  As we discussed in Remark \ref{rem:coefficients}, the
category $\gls{sfB}$ has relations which are independent of $p$.
This allows us to compare the representations of $\EuScript{A}_\phi$,
by matching them with the representations of $\mathsf{B}.$ 


\begin{definition}\label{def:Lambda}
  Let $\gls{Lambda}\subset \Z^{d}$ be the vectors such that $\rACp_{\Ba}$
  contains $\xi_{1/p}$ for $\xi\in \gls{ft1}_{1,\Z}$.  Note that this set
  makes sense for an arbitrary $\phi \in \R^{d}$.  Let $\gls{barLambda}$ be the quotient of this set by
  the action of $\gls{What}$ on $\Lambda$.
\end{definition}
The set $\bar \Lambda$ is finite; its size is bounded above by the number
of collections of weights $\vp_i$ for $i\in \weights_{\upsilon'}$ which form a basis of $\ft^*$.   We
can then divide up choices of $\phi$ according to what the
corresponding set $\bar \Lambda$ is.  Given  $\Ba\in \Z^d$, we let
$\bar{\Ba}$ be its $\widehat{W}$-orbit.
\begin{definition}\label{def:BLam}
  For a fixed $\gls{barLambda}$, we let $\glslink{BLam}{\mathsf{B}^{\bar \Lambda}(\K)}$
  for any commutative ring $\K$ be the category with object set
  $\bar \Lambda$ such that
  $\Hom_{\mathsf{B}^{\bar
      \Lambda}(\K)}(\bar{\Ba},\bar{\Bb})=\Hom_{\mathsf{B}(\K)}(\eta_\Ba,\eta_\Bb)$,
  for $\eta_\Ba$ an arbitrary element of the chamber $\rACp_{\Ba}$.

  We can also encapsulate this in the ring
  \begin{equation}
  \gls{A}(\K)=\bigoplus_{\bar{\Ba},\bar{\Bb}\in
    \bar\Lambda^{\R}}\Hom_{\glslink{BLam}{\mathsf{B}^{\bar{\Lambda}^{\R}}(\K)}}(\bar{\Ba},\bar{\Bb}),\label{eq:A-def}
\end{equation}

\end{definition}
Note that this is
equivalent to $\gls{sfA}(\K)$.  

Since the center $Z(A(\K))$ is of finite
codimension, the algebra $A(\K)$ has finitely many graded simple
modules, all of which are finite dimensional.  Each such simple for
$\K=\Q$ has a $\Z$-form, which remains irreducible mod $p$ for all but finitely
many $p$.  That is:
\begin{theorem}
For a fixed  $\bar
\Lambda\subset \Z^d/\gls{What}$, and  $p\gg 0$, there is a bijection
between simples $L$ over $\mathsf{B}^{\bar \Lambda}(\Q)$  and simples $L(p)$ over $\EuScript{A}_{\phi}$ for all $\gls{flav}\in \Z^d$ with $\bar
\Lambda(\flav)=\bar \Lambda$.   Under this bijection,  each weight space of
$L(p)$ for a weight $\upsilon$ with $\upsilon_{1/p}\in \rACp_{\Ba}$ is the same
dimension as the $\Q$-vector space $L(\Ba)$.  
\end{theorem}
\begin{proof}
  As discussed above, for $p\gg 0$, the simple graded representations
  of $\mathsf{B}^{\bar \Lambda}({\mathbb{F}_p})$ are given by reductions mod $p$ of an
  arbitrary invariant lattice of the
  simples $L$ of $\mathsf{B}^{\bar \Lambda}(\Q)$.  This clearly preserves the dimension
  of the vector space assigned to an object $\Ba$.  Under the
  equivalence of Theorem \ref{thm:pStein-equiv}, the $\upsilon$ weight space of
  a $\EuScript{A}_{\phi}$-module matches the vector space assigned
  $\Ba$ defined as before in the $\mathsf{B}^{\bar \Lambda}({\mathbb{F}_p})$-module.  
\end{proof}
Thus, the dimension of $L(p)$ only depends on the number of weights of
$\EuScript{A}_{\phi}$  with $\upsilon_{1/p}\in \rACp_{\Ba}$: it is the sum of the
dimensions of the 
spaces $L(\Ba)$ weighted by this count of integral points in a
polytope.  By the usual quasi-polynomiality of Erhart polynomials, we have that:
\begin{corollary} Fix $\gls{barLambda}$ and  let $p$ and $\phi$ vary over values where $\bar{\Lambda}(\phi)=\bar{\Lambda}$.  
  For $p\gg 0$, the dimension of $L(p)$ is a quasi-polynomial function of $\gls{flav}/p$ and
  $p$.  
\end{corollary}

%%% Local Variables:
%%% mode: latex
%%% TeX-master: "coherent-coulomb"
%%% End:
