\section{Slodowy slices in type A}

One particularly interesting special case of the constructions we have discussed are the {\bf S3 varieties for $\mathfrak{sl}_n$}.  These are resolutions of the intersections of Slodowy slices and nilpotent orbits.  Every one of these varieties can be written as a Nakajima quiver variety and as an affine Grassmannian slice (both in type $A$).  That is, they have a realization both as Higgs and as Coulomb branches of quiver gauge theories.  

Let us remind the reader of the combinatorics underlying this realization.  Given a partition $\lambda=(\la_1\geq \la_2\geq \cdots)$ of $N$ with $n$ parts, we can consider $\lambda$ as a (co)weight of $\mathfrak{sl}_n$, in the usual way.  Given $\mu$, another partition of $N$, 
we let \[\glslink{Bw}{w_i}=\lambda_i-\lambda_{i+1}\qquad \glslink{Bv}{v_i}=\sum_{k=1}^i \la_k-\mu_k.\]
The significance of these are more easily seen from the familiar formulae \[\la=\sum_{i=1}^n\glslink{Bw}{w_i}\omega_i\qquad \mu=\la-\sum_{i=1}^n\glslink{Bv}{v_i}\al_i.\]
\begin{theorem}[\mbox{\cite[Th. 1.2]{MV08}, \cite[Th. 5.6]{BFNline}}]
  The S3 variety $\mathfrak{X}^\la_\mu$ given by the slice to nilpotent matrices of Jordan type $\mu$ in the closure of those of Jordan type $\la$ is isomorphic to the affine Grassmannian slice to $\operatorname{Gr}^{\mu}$ inside $\operatorname{Gr}^{\bar \la}$, that is, to the Coulomb branch $
  \gls{Coulomb}$ of the quiver gauge theory with dimension vectors $\gls{Bw}$ and $\gls{Bv}$.   The resolution $\gls{tM}$ attached to a cocharacter $\xi \colon \C^*\to G$ is isomorphic to the convolution resolution of $\operatorname{Gr}^{\bar \la}_\mu$.  
\end{theorem}
On the other hand, the Higgs branch of this same gauge theory is the slice to nilpotent matrices of Jordan type $\la^t$  inside the closure of those of Jordan type  $\mu^t$. 

Thus, we find that these S3 varieties carry a tilting generator with a particularly nice structure:
\begin{theorem}
  For each choice of positions $\boldsymbol{\theta}$ of red strands, and labeling of red strands with fundamental weights, with $\omega_i$ appearing $\glslink{Bw}{w_i}$ times,
  the variety $\mathfrak{X}^\la_\mu$ carries a tilting generator whose endomorphism algebra is a cylindrical KLR algebra of type $A$ with this positioning of red strands, and $\glslink{Bv}{v_i}$ black strands with label $i$. 
  
  The action of wall-crossing functors on coherent sheaves matches the action on modules over this KLR algebra of derived tensor product with the bimodules $\mathbb{\mathring{B}}_{s}$.
\end{theorem}

In future work, we will study the structure of this algebra using the techniques of representation theory; in particular, using the techniques of web bimodules \cite{Webweb}, we can define an action of $\mathfrak{\widehat{sl}}_\ell$ matching that discussed by Cautis and Koppensteiner \cite{CKexotic}.  The wall-crossing functors are the action of Rickard complexes in this category, which leads us both to an explicit calculation of the decategorified action of wall-crossing (and thus a hands-on proof of Conjecture \ref{conj:BO} in this case) and insight on the exotic $t$-structure Koszul dual to that induced by our tilting generator, recovering the work of Anno and Nandakumar \cite{Anno,ANexotic} on 2-row Springer fibers.
%(which are the Coulomb branches for the action of $SL_{v_1}$ on $(\C^{v_1})^{\oplus w_1}$).  

\subsection{Kleinian singularities}

The simplest special case is the Kleinian singularity $\C^2/(\Z/\ell\Z)$.  This is isomorphic to the slice to the subregular orbit of $\mathfrak{sl}_\ell$ in the full nilcone, i.e. Jordan types $\la=(\ell,0)$ and $\mu=(\ell-1,1)$.  Thus, this corresponds to the case where $w_1=\ell$ and $v_1=1$.  That is, we have $\ell$ red strands with the same label and a single black strand.

We have $\ell$ different idempotents depending on the position of the black strand, which we think of as positioned cyclically on a circle.  The algebra of endomorphisms is generated by these idempotents, and by the degree 1 maps joining adjacent chambers by crossing the red strand:
\begin{equation*}
    \tikz{      \node at (2.5,0){ 
        \tikz[very thick,xscale=1]{
          \draw[fringe] (-1.7,-.5)-- (-1.7,.5);
          \draw[fringe] (1.7,.5)-- (1.7,-.5);
           \draw[wei] (-1,-.5)-- (-1,.5);
          \draw[wei] (.3,-.5)-- (.3,.5);
          \draw[wei] (1.5 ,-.5)-- (1.5,.5);
\draw (.9 ,-.5)  to[out=90,in=-90] (-.2,.5);
       }
      };
      \node at (9,0){ 
        \tikz[very thick,xscale=1, yscale=-1]{
           \draw[fringe] (-1.7,.5)-- (-1.7,-.5);
          \draw[fringe] (1.7,-.5)-- (1.7,.5);
         \draw[wei] (-1,-.5)-- (-1,.5);
          \draw[wei] (.3,-.5)-- (.3,.5);
          \draw[wei] (1.5 ,-.5)-- (1.5,.5);
\draw (.9 ,-.5)  to[out=90,in=-90] (-.2,.5);
       }
      };
      }
\end{equation*}
Thus, this algebra can be written as a quotient of the path algebra of the quiver with $\ell$ cyclically ordered nodes and edges joining adjacent pairs of edges in both directions.  The only relations needed are that the two length two paths starting and ending at a given node are equal: they are both multiplication by a single dot on the single black strand by (\ref{w-cost-1}).
Example \ref{example:NZ} covers the $n=2$ case.  In the $n=3$ case, we have the quiver shown below, with the diagrams above corresponding to a single pair of edges (with the others coming from rotations of these diagrams).
\[\tikz[very thick,scale=1.8]{
\node[circle,fill=black, inner sep=3pt,outer sep=2pt] (a) at (0,0){};
\node[circle,fill=black, inner sep=3pt,outer sep=2pt] (b) at (-.7,1){};
\node[circle,fill=black, inner sep=3pt,outer sep=2pt] (c) at (.7,1){};
\draw[->] (a) to[out=105, in=-45] (b);
\draw[<-] (a) to[out=135, in=-75] (b);
\draw[->] (a) to[out=45, in=-105] (c);
\draw[<-] (a) to[out=75, in=-135] (c);
\draw[->] (c) to[out=165, in=15] (b);
\draw[<-] (c) to[out=-165, in=-15] (b);
}\]

\subsection{2-row Slodowy slices}

Another case which has attracted considerable attention is that of 2-row Slodowy slices.  That is, for $k\leq \ell/2$, we consider the case $\la=(\ell,0)$ and $\mu=(\ell-k,k)$.  Thus, we have $w_1=\ell,v_1=k$.  The result is a cylindrical version of the algebras $\tilde{T}^\ell_k$ defined in \cite[Def. 2.3]{WebTGK}.

In addition to the wall-crossing functors, we can construct cylindrical versions of the cup and cap functors of \cite[Def. 2.3]{WebTGK}. These give an action of affine tangles on the categories of coherent sheaves on these varieties with $\ell-2k$ held constant; we'll show in future work that this agrees with that of Anno and Nandakumar \cite{ANexotic}.

\subsection{Cotangent bundles to projective spaces}
The example of $T^*\mathbb{P}^n$  corresponds to thinking of this as the S3 variety for the minimal orbit in type A, that is, for the Jordan types $\la=(2,1,\dots, 1,0)$ and $\mu=(1,\dots, 1)$.  This corresponds to the quiver gauge theory attached to a linear quiver $n-1$ nodes, and vectors $\gls{Bw}=(1,0,\cdots,0,1)$ and $\gls{Bv}=(1,\dots, 1)$. 
One can easily check that the associated representation is that of 
 $G=D\cap SL_n$, the diagonal matrices of determinant $1$ acting on $V=\C^n$.  
 
 
 The lattice $\ft_{\K}$ is thus just the elements of $\Z^n$ whose entries sum to 0.  We have an isomorphism between the Coulomb branch and the functions on this variety sending \[\phi_i\mapsto x_i\frac{\partial}{\partial x_i}\qquad r_{\nu}\mapsto \prod x_i^{\max(\nu_i,0)}\Big(\frac{\partial}{\partial x_i}\Big)^{\max(-\nu_i,0)}.\]


Thus, the corresponding cylindrical KLR algebra has a single black strand labeled by each of $1,\dots, n-1$, the nodes of the quiver, and red strands labeled by $1$ and $n-1$.  All idempotents are isomorphic to one of the form $e_{n-k-1}=({\color{red} 1}, 1,2,3,\dots, k-1, {\color{red} n-1}, n-1, n-2, \dots,k)$ for $k=1, 1,\dots, n$.

The choice of $\phi$ is just an assignment of $\phi_1$ to the red strand with label ${\color{red} 1}$ and $\phi_2$ to the red strand with label ${\color{red} n-1}$.  Given this choice of $\phi$, each vector   $\mu=(\mu_1,\dots, \mu_{n-1})\in \K^{n-1}$ gives an associated line bundle $\mathcal{Q}_\mu$.

The different line bundles correspond to different chambers; these are easy to visualize if we change to the coordinates \[(\nu_1=\mu_1-\phi_1,\nu_2=\mu_2-\mu_1,\dots, \nu_{n-1}=\mu_{n-1}-\mu_{n-2}).\]  In these coordinates, the hyperplanes separating chambers are given by $\nu_i=0$ and \[\mu_{n-1}-\phi_1=\sum_{i=1}^{n-1}\nu_i=\phi_2-\phi_1.\]  Thus, we can picture an $n-1$ dimensional cube, sliced by $p$ hyperplanes separating it into chambers.  If we choose the lifts of $\nu_i$ and $\phi_2-\phi_1$ in $\{0,\dots, p-1\}$,  then $e_{1}$ corresponds to elements with sum $<\phi_2-\phi_1$, $e_{2}$ to elements with $\phi_2-\phi_1\leq \sum_{i=1}^{n-1}\nu_i< \phi_2-\phi_1+p$, and so on.  


Consider $M_k=\oplus_{m}e_1({}_{m}T_{0})e_k$; the elements of this are diagrams with $e_k$ at the bottom, $e_1$ at the top, with the red line with label $n-1$ staying in place, and that with label $1$ wrapping $m$ times in the clockwise direction.  We consider this as a graded module over the homogeneous coordinate ring, identified with $\oplus_{m}e_1({}_{m}T_{0})e_1$.

This isomorphism with the coordinate ring of $T^*\mathbb{P}^{n-1}$ sends
  \begin{itemize}
  \item The diagram in ${}_{0}T_{0}$ wrapping the strands with label
    $i, i+1,\dots, i+p$ clockwise to
    $x_i\frac{\partial}{\partial x_{i+p}}$.
  \item That wrapping the same strands counterclockwise with
    $x_{i+p}\frac{\partial}{\partial x_{i}}$.
  \item The dot on the $i$th strand with
    $\sum_{m=1}^{i}x_{m}\frac{\partial}{\partial x_{m}}$.
  \item The diagram $e_1{}_{1}T_{0}e_1$ where the strands with labels
    $1,\dots, p-1$ stay straight (and thus cross the red strand with
    label $n-1$) and the strands with labels $p,\dots, n-1$ wrap
    around clockwise (and thus cross the red strand with label $1$)
    correspond to the sections $x_p$ of $\mathcal{O}(1)$.
  \end{itemize}
  \begin{equation*}
        \tikz[xscale=.9]{
      \node[label=below:{$ x_i\frac{\partial}{\partial x_{i+p}}$}] at (-4.5,0){ 
       \tikz[very thick,scale=2.5]{
          \draw[fringe] (-.7,-.5)-- (-.7,.5);
          \draw[fringe] (1.7,.5)-- (1.7,-.5);
          \draw[wei] (-.5,-.5) to node[scale=.8,below,at start,red]{$1$} (-.5,.5);
          \draw[wei] (1.5 ,-.5) to node[scale=.8,below,at start,red]{$n-1$}(1.5,.5);
           \draw (1.35 ,-.5) to[out=90,in=-90] node[scale=.8,above, at end]{$n-1$}(1.35,.5);
              \draw (1 ,-.5) to[out=90,in=-90] node[scale=.8,below, at start]{$i+p+1$}(1,.5);
\draw (-.7,-.2) to[out=0,in=-135] node[scale=.8,above, at end]{$i+p$}(.7,.5);
           \draw (.7 ,-.5) to[out=45,in=180] (1.7,-.2);
           \draw (-.7,.2) to[out=0,in=-135] (.3,.5);
           \draw (.3 ,-.5) to[out=45,in=180]node[scale=.8,below, at start]{$i$} (1.7,.2);
              \draw (-.35 ,-.5) to[out=90,in=-90] node[scale=.8,above, at end]{$1$}(-.35,.5);   
              \draw (0 ,-.5) to[out=90,in=-90]node[scale=.8,above, at end]{$i-1$} (0,.5);
              \node at (.6,-.4) {$\cdots$};
            \node at (.4,.4) {$\cdots$};
              \node at (1.175,-.4) {$\cdots$};
            \node at (1.175,.4) {$\cdots$};
              \node at (-.175,-.4) {$\cdots$};
            \node at (-.175,.4) {$\cdots$};
        }
      };
            \node[label=below:{$ x_{i+p}\frac{\partial}{\partial x_{i}}$}] at (4.5,0){ 
       \tikz[very thick,xscale=-2.5,yscale=2.5]{
          \draw[fringe] (-.7,-.5)-- (-.7,.5);
          \draw[fringe] (1.7,.5)-- (1.7,-.5);
          \draw[wei] (-.5,-.5) to node[scale=.8,below,at start,red]{$n-1$} (-.5,.5);
          \draw[wei] (1.5 ,-.5) to node[scale=.8,below,at start,red]{$1$}(1.5,.5);
           \draw (1.35 ,-.5) to[out=90,in=-90] node[scale=.8,above, at end]{$1$}(1.35,.5);
              \draw (1 ,-.5) to[out=90,in=-90] node[scale=.8,above, at end]{$i-1$}(1,.5);
\draw (-.7,-.2) to[out=0,in=-135] (.7,.5);
           \draw (.7 ,-.5) to[out=45,in=180] node[scale=.8,below, at start]{$i$}(1.7,-.2);
           \draw (-.7,.2) to[out=0,in=-135] node[scale=.8,above, at end]{$i+p$}(.3,.5);
           \draw (.3 ,-.5) to[out=45,in=180] (1.7,.2);
              \draw (-.35 ,-.5) to[out=90,in=-90] node[scale=.8,above, at end]{$n-1$}(-.35,.5);   
              \draw (0 ,-.5) to[out=90,in=-90]node[scale=.8,below, at start]{$i+p+1$} (0,.5);
              \node at (.6,-.4) {$\cdots$};
            \node at (.4,.4) {$\cdots$};
              \node at (1.175,-.4) {$\cdots$};
            \node at (1.175,.4) {$\cdots$};
              \node at (-.175,-.4) {$\cdots$};
            \node at (-.175,.4) {$\cdots$};
        }
      };}
\end{equation*}
  

  Under this isomorphism, the module $M_{k+1}$ corresponds to $\mathcal{O}(k)$.
Thus, we have that the resulting tilting generator is of the form \[\mathcal{Q}_{\mu}\cong \bigoplus_{k=0}^{n-1}\mathcal{O}(k)^{\oplus a_k},\] where $a_k$ is the number of elements of $\ft_{\Fp}$ in the corresponding chamber;  it is well-known that this gives us a tilting generator if and only if all $a_k>0$.

\subsection{The noncommutative Springer resolution}
\label{sec:nonc-spring-resol}

One final variety of considerable interest that appears here is the cotangent bundle to the type A flag variety $T^*GL_n/B$.  This arises from the dimension vectors $\gls{Bv}=(1,2,3,\dots, n-1)$ and $\gls{Bw}=(0,\cdots, 0, n)$.  In this case, $\mu=\la^t=(1,\dots, 1)$ and $\la=\mu^t=(n)$.   It is a well-known theorem of Nakajima that the Higgs branch of this theory is $T^*G/B$, and the equality $\mu=\la^t$ shows that this example is self-dual, and the Coulomb branch arises the same way.

It's also well-known that the quantum Coulomb branch that arises this way is essentially the universal enveloping algebra of $\mathfrak{gl}_n$; if we fix the flavors to numerical values, then this is the quotient of this ring by a maximal ideal of its center, but keeping the flavors as variables, it's easy to construct $U(\mathfrak{gl}_n)$ on the nose.

The construction of a tilting generator in Section \ref{sec:geometry} is thus just a rephrasing of the noncommutative Springer resolution as constructed by Bezrukavnikov, \Mirkovic, Rumynin and Riche \cite{BMRR, BezNon}; the integrable system we consider is precisely the Gelfand-Tsetlin system as discussed in \cite{WWY}.  Thus, while we obtain a familiar object, we obtain a new perspective on it, since this KLR presentation is not at all obvious from the Lie theoretic perspective.  Developing its consequences will have to wait for future work.


%%% Local Variables:
%%% mode: latex
%%% TeX-master: "coherent-coulombII"
%%% End:
